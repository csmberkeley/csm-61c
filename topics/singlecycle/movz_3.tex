\begin{blocksection}

\question
Generate the control signals for movz. The values should be 0, 1, or X (don’t care) terms. You must use don’t care terms where possible.

\begin{tabular}{ |l|l|l|l|l|l|l|l|l|l| } 
 \hline
 MOVZ & PCSel & ImmSel & RegWEn & BrUn & BSel & ASel & ALUSel & MemRW & WBSel \\ [10pt]
 \hline
 1 & & & & & & & & & \\ [10pt]
 \hline
\end{tabular}

    
\begin{solution}[0.5in]
 \begin{tabular}{ |l|l|l|l|l|l|l|l|l|l| } 
 \hline
 MOVZ & PCSel & ImmSel & RegWEn & BrUn & BSel & ASel & ALUSel & MemRW & WBSel \\ [10pt]
 \hline
 1 & 0 & X & 0 & X & 0 & 0 & OR, ADD, or SUB & 0 & X \\ [10pt]
 \hline
\end{tabular}
Note: Our answers to 4.2b and 4.2c nullify our need for 4.2a and ALUSel in this question. However, we leave the original exam solutions here for consistency.
\end{solution}

\end{blocksection}