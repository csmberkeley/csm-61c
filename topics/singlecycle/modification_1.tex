\begin{blocksection}
\question
We want to expand our instruction set from the base RISC-­V ISA (RV32I) to support some new instructions. You can find the canonical single-cycle datapath above. For the proposed instruction below, choose ONE of the options below.

\begin{enumerate}
    \item Can be implemented without changing datapath wiring, only changes in control signals are needed. (i.e. change existing control signals to recognize the new instruction)
    \item Can be implemented, but needs changes in datapath wiring, only additional wiring, logical gates and muxes are needed.
    \item Can be implemented, but needs change in datapath wiring, and additional arithmetic units are needed (e.g. comparators, adders, shifters etc.).
    \item Cannot be implemented.
\end{enumerate}

(Note that the options from 1 to 3 gradually add complexity; thus, selecting 2 implies that 1 is not sufficient. You should select the option that changes the datapath the least (e.g. do not select 3 if 2 is sufficient). You can assume that necessary changes in the control signals will be made if the datapath wiring is changed.)

\begin{parts}

\item Allowing software to deal with 2’s complement is very prone to error. Instead, we want to implement the negate instruction, \texttt{neg rd rs1}, which puts \texttt{-R[rs1]} in \texttt{R[rd]}.

\begin{solution}[0.5in]
A. This is a tricky question! Notice \texttt{neg} doesn’t use all available bits, so we could make \texttt{neg rd, rs1} into a special R-type instruction \texttt{neg rd, x0, rs1} such that the instruction does \texttt{R[rd] = x0 - R[rs1]}. Notice that subtraction is supported by our default datapath. So, we only need to add the new control signal neg which will produce the same \texttt{ALUsel}, \texttt{Asel}, \texttt{Bsel}, … signals a \texttt{sub} does.
\end{solution}

\item Sometimes, it is necessary to allow a program to self­-destruct. Implement \texttt{segfault rs2, offset(rs1)}. This instruction compares the value in \texttt{R[rs2]} and the value in \texttt{MEM[R[rs1]+offset]}. If the two values are equal, write \texttt{0} into the \texttt{PC}; otherwise treat this instruction as a \texttt{NOP}.

\begin{solution}[0.5in]
 C. Need to 1) Add a comparator after memory unit and wire the output to \texttt{PCSel}. 2) Add a zero wired to \texttt{mux} before \texttt{PC}. 3) Change corresponding \texttt{PCSel} signal width.
\end{solution}
\end{parts}

\end{blocksection}