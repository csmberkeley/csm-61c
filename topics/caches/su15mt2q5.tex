\begin{blocksection}
\question
Assume we are working in a 32-bit physical address space. We have two possible data caches: cache X is a direct-mapped cache, while cache Y is 2-way associative with LRU replacement policy. Both are 4 KiB caches with 512 B blocks and use write-back and write-allocate policies. 

Use the code below to answer part b and c.

\begin{verbatim}
int DOUBLE_ARRAY_SIZE = 2 * 1024; double_arr[DOUBLE_ARRAY_SIZE]; 
for (int i = 0; i < DOUBLE_ARRAY_SIZE; i++) /* loop 1 */        
    double_arr[i] = i; 
for (int i =0;i < DOUBLE_ARRAY_SIZE; i += 8) /* loop 2 */       
    double_arr[i] *= double_arr[0];
\end{verbatim}

\begin{parts}

\part
Calculate the number of bits used for Tag, Index and Offset for cache X and Y.
\begin{solution}[0.5in]
Cache X: 20 tag, 3 index, 9 offset 
Cache Y: 21 tag, 2 index, 9 offset
\end{solution}

\part
What is the hit rate for each cache if we run only loop 1? (hint: they’re both the same). What types of misses do we get?
\begin{solution}[0.5in]
Both have a hit rate of 63/64. Compulsory
\end{solution}

\part
What is the hit rate of each cache when you execute loop 2? Assume that you have executed loop 1. Assume the worst case ordering of accesses within a single iteration of the loop if multiple orders are possible. You may leave your answer as an expression involving products and sums of fractions.
\begin{solution}[0.5in]
Cache X has a hit rate of 29/32 * 23/24 + 3/32 * 7/24.
Cache Y has a hit rate of 23/24.
\end{solution}

\part
Compute the AMAT for the following system with 3 levels of caches. (You should not need any information from the previous parts of this problem.) Give your answer as a decimal value.
\begin{center}
\begin{tabular}{ |c|c|c|c| } 
\hline
L1 Cache & L2 Cache & L3 Cache & Main Memory \\ 
\hline
Hit time: 1 ns & Hit time: 5 ns & Hit time: 15 ns & Hit time: 500 ns \\ 
Global miss rate: 50\% & Local miss rate: 20\% & Local miss rate: 1\% & \\
\hline
\end{tabular}
\end{center}
\begin{solution}
1 + .5*(5 + .20*(15 + .01*(500))) = 5.5 ns
\end{solution}

\end{parts}

\end{blocksection}