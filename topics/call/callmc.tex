\begin{blocksection}
\question The following are some multiple choice questions about CALL. Clearly circle the correct answer: 

\begin{parts}
\item 
A system program that combines separately compiled modules of a program into a form suitable for execution is $\rule{1cm}{0.15mm}$. \newline
A. Assembler \newline
B. Loader \newline
C. Linker \newline
D. None of the Above 

\begin{solution}
    C
\end{solution}

\item
At which point will all the machine code bits be determined for a $la$ instruction? \newline
A. C code \newline
B. Assembly code \newline
C. Object code \newline
D. Executable 

\begin{solution}
    D
\end{solution}

\item
At the end of the assembling stage, the symbol table contains the $\rule{1cm}{0.15mm}$ of each symbol. \newline
A. relative address \newline
B. absolute address \newline
C. the stack segment beginning address \newline
D. the global segment beginning address 

\begin{solution}
    A
\end{solution}

\item
beq and bne instructions produce $\rule{1cm}{0.15mm}$ and they $\rule{1cm}{0.15mm}$. \newline
A. PC-relative addressing, never relocate \newline
B. PC-relative addressing, always relocate \newline
C. Absolute addressing, never relocate \newline
D. Absolute addressing, always relocate 

\begin{solution}
    A
\end{solution}

\item
jal and jalr instructions add symbols and $\rule{1cm}{0.15mm}$ to $\rule{1cm}{0.15mm}$. \newline
A. instruction addresses, the symbol table \newline
B. symbol addresses, the symbol table \newline
C. instruction addresses, the relocation table \newline 
D. symbol addresses, the relocation table

\begin{solution}
    C
\end{solution}
\end{parts}
\end{blocksection}