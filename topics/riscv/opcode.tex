\begin{blocksection}
\question
As a curious 61C student, you question why there are so many possible \lstinline$opcode$s, but only 47 instructions. Thus, your propose a revision to the standard 32-bit RISC-V instruction formats where each instruction has a unique \lstinline$opcode$ (which still is 7 bits). You believe this justifies taking out the \lstinline$funct3$ field from the \lstinline$R$, \lstinline$I$, \lstinline$S$, and \lstinline$SB$ instructions, allowing you to allocate bits to other instruction fields except the \lstinline$opcode$ field. 

\begin{parts}
\part
What is the largest number of registers that can now be supported in hardware?

\begin{solution}[0.5in]
64
\end{solution}

\part
With the new register size, how far can a \lstinline$jal$ instruction jump to (in halfwords)?

\begin{solution}[0.5in]
$[-2^{18}, 2^{18-1} ]$
\end{solution}

\part
Assume register \lstinline$s0 = 0x1000 0000$, \lstinline$s1 = 0x4000 0000$, \lstinline$PC = 0xA000 0000$. Let’s analyze the instruction \lstinline$jalr s0, s1, MAX_POS_IMM$ where \lstinline$MAX_POS_IMM$ is the maximum possible positive immediate for \lstinline$jalr$. Using the register sizes defined above, what are the values in registers \lstinline$s0$, \lstinline$s1$, and \lstinline$pc$ after the instruction executes?

\begin{solution}[0.5in]
\begin{verbatim}
s0 = 0xA000 0004
s1 = 0x4000 0000
pc = 0x4000 0FFF
\end{verbatim}
\end{solution}
\end{parts}

\end{blocksection}