% This is supposed to be used in conjunction with reverse.tex
% If do not want the C code in reverse.tex, add \question at the beginning of the blocksection environment
\begin{blocksection}
Now assume \lstinline$a0$ contains the address of the head of a linked list.  Fill in the function below to reverse a linked list. The function reverse follows calling conventions and does not return anything. You may not need all lines. 

\begin{verbatim}
1.  reverse: ______________
2.  ______________
3.  ______________
4.  ______________
5.  add s0 a0 x0         # s0 corresponds to curr
6.  xor s2 s2 s2         # s2 corresponds to prev
7.  loop: ___ s0 x0 exit
8.  ______________
9.  ______________
10. add s2 s0 x0 
11. add s0 s1 x0 
12. j loop 
13. exit: ______________
14. ______________
15. ______________
16. addi sp sp 12
17. jr ra
\end{verbatim}

\begin{solution}[4in]
At the beginning and end of a RISC-V function, we need to add a prologue (lines 1-4) and an epilogue (lines 13-15), as the saved registers must be preserved when we call reverse, and saved registers \lstinline$s0$, \lstinline$s1$, and \lstinline$s2$ are all being modified and must be saved on the stack.

Also note that \lstinline$curr->next$ is located 4 bytes above \lstinline$curr->val$ in the struct, per C99 standard.

\begin{verbatim}
1.  reverse: addi sp sp -12
2.  sw s0 0(sp) 
3.  sw s1 4(sp) 
4.  sw s2 8(sp) 

7.  beq s0 x0 exit    # Break if curr == NULL
8.  lw s1 4(s0)       # Write curr->next into next
9.  sw s2 4(s0)       # Write prev into curr->next

13. exit:lw s0 0(sp)
14. lw s1 4(sp)
15. lw s2 8(sp)
\end{verbatim}


\end{solution}
\end{blocksection}