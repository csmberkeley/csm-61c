\begin{blocksection}
\question
You wish to speed up one of your programs by implementing it directly in assembly. Your partner started translating the function \lstinline$is_substr()$ from C to RISC-V, but didn’t finish. Please complete the translation by filling in the lines below with RISC-V assembly. The prologue and epilogue have been written correctly but are not shown.

Note: \lstinline$strlen()$, both as a C function and RISC-V procedure, takes in one string as an argument and returns the length of the string (not including the null terminator).

\begin{verbatim}
/* Returns 1 if s2 is a substring of s1, and 0 otherwise. */
int is_substr(char* s1, char* s2) {
    int len1 = strlen(s1);
    int len2 = strlen(s2);
    int offset = len1 - len2;
    while (offset >= 0) {
        int i = 0;
        while (s1[i + offset]  == s2[i]) {
            i += 1;
            if (s2[i] == ‘\0’) 
                return 1;
        }
        offset -= 1;
    }
    return 0;
}
\end{verbatim}
\end{blocksection}

\begin{blocksection}
Fill in the following RISC-V code based on the given C code:

\begin{verbatim}
1.  is_substr:
2.      mv s1, a0
3.      mv s2, a1
4.      jal ra, strlen
5.      mv s3, a0
6.      mv a0, s2
7.      jal ra, strlen
8.      sub s3, s3, a0
9.  Outer_Loop:
10.     ______ ______, ______, False
11.     add t0, x0, x0
12. Inner_Loop:
13.     add t1, t0, s3
14.     add t1, s1, t1
15.     lbu t1, 0(t1)
16.     ____________
17.     ____________
18.     ______ t1, ______, Update_Offset
19.     addi t0, t0, 1
20.     add t2, t0, s2
21.     ____________
22.     beq t2, ______, ______,
23.     jal x0 Inner_Loop
24. Update_Offset: 
25.     addi s3, s3, -1
26.     ____________
27. False: 
28.     xor a0, a0, ______
29.     jal x0, End
30. True: 
31.     addi a0, x0, 1
32. End: ______
\end{verbatim}

\end{blocksection}

\begin{blocksection}
\begin{solution}

We shall first establish the relationship between the registers used and the symbols in the C code: s1 corresponds to s1, s2 to s2, s3 to offset starting from line 8, t0 to i, t1 to s1[i + offset], and t2 to s2[i]. We then complete the translation.

\begin{verbatim}
10.    blt s3, x0, False         # Break if offset < 0
16.    add t2 s2 t0              # Access s2[i]
17.    lbu t2 0(t2)              # Access s2[i]
18.    bne t1, t2, Update_Offset # Break if s1[i + offset] != s2[i]
21.    lbu t2 0(t2)              # Access s2[i]
22.    beq t2, x0, True          # Return 1 if s2[i] == 0
26.    jal x0 Outer_Loop         # Continue looping
28.    xor a0, a0, a0            # Set a0 = 0 for return
32. End: ret                     # Return
\end{verbatim}
\end{solution}



\end{blocksection}