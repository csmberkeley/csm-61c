\begin{blocksection}
\question
You wish to speed up one of your programs by implementing it directly in assembly. Your partner started translating the function \lstinline$is_substr()$ from C to RISC-V, but didn’t finish. Please complete the translation by filling in the lines below with RISC-V assembly. The prologue and epilogue have been written correctly but are not shown.

Note: \lstinline$strlen()$, both as a C function and RISC-V procedure, takes in one string as an argument and returns the length of the string (not including the null terminator).

\begin{verbatim}
/* Returns 1 if s2 is a substring of s1, and 0 otherwise. */
int is_substr(char* s1, char* s2) {
    int len1 = strlen(s1);
    int len2 = strlen(s2);
    int offset = len1 - len2;
    while (offset >= 0) {
        int i = 0;
        while (s1[i + offset]  == s2[i]) {
            i += 1;
            if (s2[i] == ‘\0’) 
                return 1;
        }
        offset -= 1;
    }
    return 0;
}
\end{verbatim}

Fill in the following RISC-V code based on the given C code:

\begin{verbatim}
1. is _substr:
2. 	mv s1, a0
3. 	mv s2, a1
4. 	jal ra, strlen
5. 	mv s3, a0
6. 	mv a0, s2
7. 	jal ra, strlen
8. 	sub s3, s3, a0
9. Outer_Loop:
10. ______, ______, ______, False
11. add t0, x0, x0
12. Inner_Loop:
13. add t1, t0, s3
14. add t1, s1, t1
15. lbu t1, 0(t1)
16. ____________
17. ____________
18. ______, t1, ______, Update_Offset
19. addi t0, t0, 1
20. add t2, t0, s2
21. ____________
22. beq t2, ______, ______,
23. jal x0 Inner_Loop
24. Update_Offset: addi s3, s3, -1
25. ____________
26. False: xor a0, a0, ______
27. jal x0, End
28. True: addi a0, x0, 1
29. End: ______.
\end{verbatim}

\begin{solution}
\begin{verbatim}
10. blt s3, x0, False

16. add t2 s2 t0
17. lbu t2 0(t2)
18. bne t1, t2, Update Offset

21. lbu t2 0(t2)
22. beq t2, x0, True

25. jal x0 Outer_Loop
26. xor a0, a0, a0
\end{verbatim}
\end{solution}

\end{blocksection}