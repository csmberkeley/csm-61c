\begin{blocksection}
\question
For parts a and b, answer either Caller, Callee or Neither as applicable. The Caller is the function passing these values to a new call, the Callee is the function being called (ie. if we invoke a function doStuff() from main(), doStuff() is the Callee and main() is the Caller) 

\begin{parts}
\part
Whose responsibility is it to save the return address (ra) in a function call?

\begin{solution}[0.5in]
    Caller
\end{solution}

\part
Whose responsibility is it to save the temporary registers (t0-t6)? What about the saved registers (s0-s11)?

\begin{solution}[0.5in]
    Temp: Caller
    Saved: Callee
\end{solution}

\part
You’re given a new RISC-V instruction set where instructions are 12 bits long instead of 32 bits. Each rs and rd field uses 2 bits. How many registers does this new format support? 
\begin{solution}[0.5in]
    In regular RISC-V, we can represent 32 registers because our 5-bits for each rd/rs field allow us $2^{5}$ possibilities. Here, we have only 2 bits for each rd/rs field and so we can represent $2^{2}$ = 4 registers
\end{solution}

\part
If our opcode and funct3 are now two bits each, what is the largest immediate our I-Type instruction format can support? How far can we now branch (in bytes)?

\begin{solution}[0.5in]
    For the I-Type: opcode + funct code + 2(reg field) = 2 + 2 + 2(2) = 8 bits already in use. This leaves 4 bits for our immediate with which we can represent from -8 to 7.
    For the B-type: opcode + funct code + 2(reg field) = 8 bits in use. Then there are 4 bits for our immediate. Recall that this immediate is then multiplied by 2 to determine how many bytes we are branching by, so anywhere from -16 to 14 bytes.
\end{solution}

\part
Draw the format of the immediate instruction in the following box. Label each field and the number of bits each field holds. 

\begin{solution}[0.5in]
    Immediate (4)       $\vert$        r1 (2)  $\vert$      funct (2) $\vert$      rd (2) $\vert$     opcode (2)
\end{solution}
\end{parts}
\end{blocksection}