\begin{blocksection}
\question
Answer either Caller, Callee or Neither as applicable. The Caller is the function passing these values to a new call, and the Callee is the function being called. That is, if we invoke a function doStuff() from main(), doStuff() is the Callee and main() is the Caller. If you are stuck, the RISC-V green sheet might be a great resource.

\begin{parts}
\part
Whose responsibility is it to save the return address (\lstinline$ra$) in a function call?

\begin{solution}[0.5in]
Register \lstinline$ra$ is a caller-saved register.
\end{solution}

\part
Whose responsibility is it to save the temporary registers (\lstinline$t0$-\lstinline$t6$)? What about the saved registers (\lstinline$s0$-\lstinline$s11$)?

\begin{solution}[0.5in]
Temporary registers are caller-saved, whereas saved registers are callee-saved.
\end{solution}

\part
Whose responsibility is it to save the argument registers (\lstinline$a0$-\lstinline$a7$)? Why?

\begin{solution}[0.5in]
Argument registers are caller-saved because the callee is typically expected to put its return value(s) in the argument registers.
\end{solution}

\end{parts}
\end{blocksection}

\begin{blocksection}

\question 
Look at the code segment below and identify if each function has a caller-saved register error, callee-saved register error or no error. The actual function of the code is not important. 

\begin{verbatim}
give_you_up:	addi t0, x0, 69
				slli t0, t0, 2
				add  a1, x0, t0
				addi sp, sp, -4
				sw t0, 0(sp)
				jal let_you_down
				lw t0, 0(sp)
				addi sp, sp, 4
				sw a0 0(t0)
				ecall #exit the program
				
let_you_down:	addi sp, sp, -4
				sw ra, 0(sp)
				addi t7, a1, 420
				jal desert_you
				add a0, a0, t7
				lw ra, 0(sp)
				addi sp, sp, 4
				ret
				
desert_you:		addi sp, sp -12
				sw s0, 0(sp)
				sw t0, 4(sp)
				sw ra, 8(sp)
				mv s0, a1
				srai s0, s0, 2
				addi t0, s0, 64
				lw t0, 0(t0)
				mv a0, t0
				lw s0, 0(sp)
				lw t0, 4(sp)
				lw ra, 8(sp)
				ret 
\end{verbatim}

\end{blocksection}
\begin{blocksection}

\begin{parts}
\part
What calling convention errors are present in \lstinline$give_you_up$?
\begin{solution}[0.5in]
No error. The function correctly saves the t0 register before calling \lstinline$let_you_down$. There is no need to save the save and return registers since the program exits instead of returning to another function. 
\end{solution}

\part
What calling convention errors are present in \lstinline$let_you_down$?
\begin{solution}[0.5in]
Caller-saved register error. Note that this function is both a caller and a callee. The callee saved registers are safe, but the \lstinline$t7$ register, which is a caller saved register, has not been saved before \lstinline$desert_you$ is called.
\end{solution}

\part
What calling convention errors are present in \lstinline$desert_you$?
\begin{solution}[0.5in]
Callee-saved register error. \lstinline$t0$ is a caller saved register, so it does not need to be saved. Additionally, the stack pointer is never incremented by 12 bytes before the function returns. 
\end{solution}

%lol the comments are getting FATTER with every semester 
%\part
%You’re given a new RISC-V instruction set where instructions are 12 bits long instead of 32 bits. Each rs and rd field uses 2 bits. How many registers does this new format support? 
%\begin{solution}[0.5in]
%    In regular RISC-V, we can represent 32 registers because our 5-bits for each rd/rs field allow us $2^{5}$ possibilities. Here, we have only 2 bits for each rd/rs field and so we can represent $2^{2}$ = 4 registers
%\end{solution}

%\part
%If our opcode and funct3 are now two bits each, what is the largest immediate our I-Type instruction format can support? How far can we now branch (in bytes)?

%\begin{solution}[0.5in]
%    For the I-Type: opcode + funct code + 2(reg field) = 2 + 2 + 2(2) = 8 bits already in use. This leaves 4 bits for our immediate with which we can represent from -8 to 7.
%    For the B-type: opcode + funct code + 2(reg field) = 8 bits in use. Then there are 4 bits for our immediate. Recall that this immediate is then multiplied by 2 to determine how many bytes we are branching by, so anywhere from -16 to 14 bytes.
%\end{solution}

%\part
%Draw the format of the immediate instruction in the following box. Label each field and the number of bits each field holds. 

%\begin{solution}[0.5in]
%    Immediate (4)       $\vert$        r1 (2)  $\vert$      funct (2) $\vert$      rd (2) $\vert$     opcode (2)
%\end{solution}
\end{parts}
\end{blocksection}