\begin{blocksection}
\question
You are given the code below, and told that you can read and write to any word of memory without error. The function mystery lives somewhere in memory, but not at address 0x0. Your system has no caches.
\begin{verbatim}
    mystery:
        la t6, loop3
    loop: addi x0, x0, 0      #nop
        lw t5, 0(t6)
        addi t5, t5, 0x80
        sw t5, 0(t6)
        addi a0, a0, -1
        bnez a0, loop
        ret
\end{verbatim}

\begin{parts}

\part
At a functional level, what does mystery(x) do when x $<$ 10?
\begin{solution}[0.5in]
Resets the first x registers.Resets register number 0 through x-1.
\end{solution}

\part
One by one, what are the values of a0 that bnez sees with mystery(13) at every iteration? List no more than 13; if it sees fewer than 13, write N/A for the rest
\begin{solution}[0.5in]
12, 11, 10, 9, 8, 7, 6, 5, 4, 3, -1, -2, -3
\end{solution}

\part
How many times is the bnez instruction seen when mystery(33) is called before it reaches ret (if it ever does)?
\begin{solution}[0.5in]
2^{32} + 10
\end{solution}

\end{parts}


\end{blocksection}
