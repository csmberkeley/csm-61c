\begin{blocksection}
\question You are given the code above, and told that you can read and write to any word of memory without error.
The function mystery lives somewhere in memory, but not at address 0x0. Your system has no caches.

\begin{verbatim}
1. mystery:
2.     la t6, loop
3. loop: addi x0, x0, 0   ##nop
4.     lw t5, 0(t6)
5.     addi t5, t5, 0x80
6.     sw t5, 0(t6)
7.     addi a0, a0, -1
8.     bnez a0, loop
9.     ret

\end{verbatim}

\end{blocksection}

\begin{blocksection}

\begin{solution}[0.5in]
What this function does (courtesy of Albert Zhan on piazza with minor edits):

1. mystery(x) entry point - where a0 = x, this is the label to jump to mystery.

2. la t6, loop - t6 register now stores the address of the "loop" (in this case, it points to an instruction in the text section of memory)

3. addi x0 x0 0 - instruction is executed. Note that addi instruction looks like 
\begin{verbatim}
_ _ _ _ _ _ _ _ _ _ _ _ + _ _ _ _ _ + 000 + _ _ _ _ _ + 0010011
\end{verbatim}
which are imm[11:0], rs1, and rd

4. lw t5, 0(t6) - loads the 32 bits of instruction into t5 register

5. addi t5, t5, 0x80 - adds 0000 0000 0000 0000 0000 0000 0100 0000 to the instruction at "loop" -> counting the bits, it adds 1 to rd (so the instruction on the first iteration becomes "addi x1, x0, 0")

6. sw t5, 0(t6) - stores the updated instruction into memory, so it modifies the instruction at "loop" -- self modifying the code.

7. addi a0, a0, -1 - decreases x by 1

8. bnez a0, loop - if $a0 == 0$, continue. Otherwise, go back to the instruction at "loop"

9. ret - jump to ra...

So it modifies the registers by modifying the code at "loop", which in turn modifies the registers x0, x1, etc.
\end{solution}
\end{blocksection}

\begin{blocksection}


\begin{parts}

\part
At a functional level, in seven words or fewer, what does mystery(x) do when $x < 10$?
\begin{solution}[0.5in]
Resets the first x registers.
\end{solution}

\part
One by one, what are the values of a0 that bnez sees with mystery(13) at every iteration? We’ve
done the first few for you. List no more than 13; if it sees fewer than 13, write N/A for the rest.

\begin{verbatim}
12, 11, __, __, __, __, __, __, __, __, __, __, __
\end{verbatim}

\begin{solution}[0.5in]
12, 11, 10, 9, 8, 7, 6, 5, 4, 3, -1, -2, -3

We’re merrily rolling along, resetting all the registers, when we reset x10 = a0! But then “addi a0,a0,-1” makes it -1 so it actually never hits the stopping “branch equal to zero” case then! So the bnez sees -1, then -2, then -3 as the resetter continues along its merry way.
\end{solution}

\end{parts}
\end{blocksection}





