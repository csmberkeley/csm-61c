\begin{blocksection}
\question Translate the following code from RISC-V to C. Assume that register x9 stores the address of an integer array, ptr, whose length is 30.

\begin{verbatim}
addi x6 x0 1
addi x10 x0 30
L1: bge x6 x10 Next 
andi x7 x6 1 
slli x8 x6 2 
add x8 x8 x9 
sw x7 0(x8) 
addi x6 x6 1 
j L1
Next:
\end{verbatim}

\begin{solution}[0.5in]
\begin{verbatim}
for (int i = 1; i < 30; i++) { 
    ptr[ i ] = i & 1; 
}
\end{verbatim}
\end{solution}

\question Translate the following code from C to RISC-V. Again, assume that register x9 stores the address of an integer array, ptr, whose length is 30.

\begin{verbatim}
int total = 0;
for (int i = 0; i < 30; i++) { 
    total += ptr[ i ]; 
}
\end{verbatim}

\begin{solution}[1.5in]
\begin{verbatim}
add x6 x0 x0
add x7 x0 x0
addi x10 x0 30
slli x11 x10 2
L2: bge x6 x11 End 
add x8 x9 x6 
lw x8 0(x8) 
add x7 x7 x8 
addi x6 x6 4 
j L2
End:
\end{verbatim}

Meta: 
Here it might be a good idea to introduce students to the concept of different register names and their purposes (s registers, a registers). (Thanks Ryan!) More to go over! XD
Remember that every line of code also has their own “address” that the Program Counter follows.

\end{solution}

\end{blocksection}