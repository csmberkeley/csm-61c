\begin{blocksection}
\question
For each line of RISC-V instruction, answer what will be the value saved into the corresponding registers or memory addresses. Assume that x8 contains a valid memory address, and at that address stores the word Mem[Reg[x8]] = 0x00000180. The instructions are executed in that order.

\begin{parts}
\part
\begin{verbatim}
lw x9, 0(x8)
\end{verbatim}
What value does x9 now store?

\begin{solution}[0.5in]
As we load the word at address Reg[x8] into register x9, we have Reg[x9] = 0x00000180.
\end{solution}

\part
\begin{verbatim}
lb x10, 1(x8)
\end{verbatim}
What value does x10 now store? 

\begin{solution}[0.5in]
Since RISC-V is little endian, the byte at address Reg[x8] + 1 would be the second least significant byte in the word at address Reg[x8] 0x01, and we sign-extend this number to get Reg[x10] = 0x00000001.
\end{solution}

\part
\begin{verbatim}
lb x11, 0(x8)    
\end{verbatim}
What value does x11 now store? 

\begin{solution}[0.5in]
Again, since RISC-V is little endian, the byte at address Reg[x8] would be the least significant byte in the word at address Reg[x8] 0x80, and we sign extend this number to get Reg[x11] = 0xFFFFFF80.
\end{solution}

\part
\begin{verbatim}
sb x11, 3(x8)
\end{verbatim}
What is the value of Mem[Reg[x8]]? 

\begin{solution}[0.5in]
We just take the least significant byte of Reg[x11] 0x80 and write it to the memory address Reg[x8] + 3. Since RISC-V is little endian, we have Mem[R[x8]] = 0x80000180.
\end{solution}

\part
\begin{verbatim}
lbu x12, 0(x8)
\end{verbatim}
What value does x12 now store? 

\begin{solution}[0.5in]
Still, the byte at address Reg[x8] would be 0x80, but this time we need not sign extend, as this is an unsigned operation, and so Reg[x12] = 0x00000080.
\end{solution}
\end{parts}
\end{blocksection}