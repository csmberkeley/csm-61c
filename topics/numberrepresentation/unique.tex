\begin{blocksection}
\question
How many unique integers can be represented in each case?

\begin{parts}
\part
Unsigned:

\begin{solution}[0.5in]
$2^n$
\end{solution}

\part
Two's Complement:

\begin{solution}[0.5in]
$2^n$
\end{solution}

\part
One's Complement:

\begin{solution}[0.5in]
$2^n - 1$
\end{solution}

\part
Bias (with bias $b$):

\begin{solution}[0.5in]
$2^n$
\end{solution}

\end{parts}

\begin{solution}
For both unsigned and two’s complement, each bit string corresponds to a different integer, so we have $2^n$ unique integers.
 
Ones’ complement is an exception: we have two zeroes. $2^n$ treats these two zeroes separately, but really the two are indistinguishable. Therefore we need to subtract one possibility to get $2^n - 1$.
 
Bias is just a shifted version of unsigned and so it can represent the same number of integers.
\end{solution}

\end{blocksection}