\begin{blocksection}
After solving your datapath bug, you decide to introduce the traditional five-stage pipeline into your processor.
You find that your unit tests with single commands work for all instructions, and write some test patterns with
multiple instructions. After running the test suite, the following cases fail. You should assume registers are
initialized to 0, the error condition is calculated in the fetch stage, and no forwarding is currently implemented.

\textbf{Case 1:} Assume the address of an array with all different values is stored in s0.


\textbf{addi t0 x0 1
\newline
\hspace{10mm} slli t1 t0 2
\newline
\hspace{10mm} add t1 s0 t1
\newline
\hspace{10mm} lw t2 0(t1)
\newline
}

\newline
Each time you run this test, there is the same incorrect output for t2. All the commands work individually on
the single-stage pipeline.

\question 
What caused the failure? (Select ONE)
\begin{checkboxes}
 \choice Control Hazard
 \choice Structural Hazard
 \choice Data Hazard
 \choice None of the Above
\end{checkboxes}

\begin{solution}[0.5in]
Data Hazard
\end{solution}

\question
How could you fix it? (Select ALL that apply)
\begin{checkboxes}
 \choice Insert a nop 3 times if you detect this specific error condition
 \choice Forward execute to write back if you detect this specific error condition
 \choice Forward execute to memory if you detect this specific error condition
 \choice Forward execute to execute if you detect this specific error condition
 \choice Flush the pipeline if you detect this specific error condition
\end{checkboxes}

\begin{solution}[0.5in]
A, D
\end{solution}

\end{blocksection}

\begin{blocksection}

\begin{solution}[0.5in]
The issue with the above code is the use of a register (aka we get the value during the decode phase) before
we have written back the value of the previous instruction. This is a data hazard as the data which we want is
not restored to the regfile. This means that we would have the current instruction in the execute phase while
we have the previous in decode. This means that the next cycle, we would have to forward the execute output
to the execute input to make sure the value is the correct, updated one. Inserting a nop when you realize this
error happens will allow the system to do the write back. The other forwards in this problem are necessary for
the given code above. Flushing the pipeline does not work as it means that we will no longer execute the
instructions which were flushed. This means we would just drop instructions which would not get the correct
value instead of just waiting till they can get the correct value.
\end{solution}


\end{blocksection}