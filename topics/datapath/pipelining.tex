\begin{blocksection}
\question
Assume you have a RISC-V processor that has the following execution times:

\begin{tabular}{ |l|l|l|l|l| } 
 \hline
 IF & ID & EX & MEM & WB \\
 \hline
 100 ps & 150 ps & 300 ps & 400 ps & 250 ps \\
 \hline
\end{tabular}

\begin{parts}
\item In an unpipelined processor, what is the maximum clock rate possible?

\begin{solution}[0.5in]
The critical path through the circuit is the entire datapath (all 5 stages).
$100 + 150 + 300 + 400 + 250 = 1200 \text{ps} \rightarrow \frac{1}{1200} s \rightarrow 833 \text{MHz}$
\end{solution}

\item In a pipelined processor, what is the maximum clock rate possible?

\begin{solution}[0.5in]
The critical path through the circuit is the longest of the stages: Memory
$\frac{1}{400} \text{ps} \rightarrow 2.5 \text{GHz}$
\end{solution}

\item Suppose we add some hardware that shortens the ALU processing time to 250 ps. Does this change the maximum clock rate in an unpipelined processor? In a pipelined processor?

\begin{solution}[0.5in]
In the unpipelined case, the clock rate will be greater because we’ve shortened the critical path. However, in the unpipelined case, the clock rate will not change because the Memory stage is still critical path.
\end{solution}

\end{parts}
\end{blocksection}