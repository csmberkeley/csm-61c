\begin{blocksection}
\textbf{Case 2:} After fixing that hazard, the following case fails:

\textbf{addi s0 x0 4
\newline
\hspace{10mm} slli t1 s0 2
\newline
\hspace{10mm} bge s0 x0 greater
\newline
\hspace{10mm} xori t1 t1 -1
\newline
\hspace{10mm} addi t1 t1 1
\newline
\hspace{10mm} greater:
\newline
\hspace{10mm} mul t0 t1 s0
\newline
}
\newline
When this test case is run, t0 contains 0xFFFFFFC0, which is not what it should have been.

\question 
What caused the failure? (Select ONE)
\begin{checkboxes}
 \choice Control Hazard
 \choice Structural Hazard
 \choice Data Hazard
 \choice None of the Above
\end{checkboxes}

\begin{solution}[0.5in]
Control Hazard
\end{solution}

\question
How could you fix it? (Select ALL that apply)
\begin{checkboxes}
 \choice Insert a nop 3 times if you detect this specific error condition
 \choice Forward execute to write back if you detect this specific error condition
 \choice Forward execute to memory if you detect this specific error condition
 \choice Forward execute to execute if you detect this specific error condition
 \choice Flush the pipeline if you detect this specific error condition
\end{checkboxes}

\begin{solution}[0.5in]
A, E
\end{solution}

\begin{solution}[0.5in]
The issue with the code above is we do not clear/flush the instructions if the branch determines it is taken.
Remember that we are running on a five stage pipeline CPU which just assumes PC + 4 unless an instruction
says otherwise. This means that we will not determine if the branch is taken until the branch is in the execute
phase. This means that we will have the next two instructions already in the pipeline (one in instruction fetch,
the other in instruction decode). So we have a Control Hazard as we are not executing the correct instructions.
Some ways how to fix it: insert nops if you detect a branch instruction in the instruction fetch stage OR flush
the pipeline if the branch is in the opposite direction of what was predicted. Forwarding data in this case will not
help at all.
\end{solution}
\end{blocksection}