\begin{blocksection}

\question
Implement the following function using openMP:
\begin{verbatim}
// Sequential code
static int selective_square_total (int n, int *a, int c) {
    for (int i = 0; i < n; i += 1) {
        if (a[i] > c) {
            a[i] *= a[i];
        }
    return product;
}   
\end{verbatim}

\begin{verbatim}
// openMP for code
static int selective_square_parallelized (int n, int *a, int c) {
    #pragma omp parallel {
    #pragma omp for
    for (int i = ___; i < ___; i += ___) {
        if (_____ > _____) {
            a[i] *= _____;
        }
    }
    return product;
}
\end{verbatim}

\begin{solution}[0.5in]
\begin{verbatim}
    for (int i =  0 ; i <  n ; i += 1) {
        if ( a[i]  >   c  ) {
            a[i] *=  a[i] ;
        }
    }
\end{verbatim}
This is a bit of a trick question: openMP takes care of the parallelism for us by using \#pragma omp parallel / \#pragma omp for.
META: Students might get confused on whether they should chunk or alternate threads, because that’s usually what the lab covers and/or what lecture covers conceptually.
\end{solution}
    
\end{blocksection}