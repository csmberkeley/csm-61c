\begin{blocksection}

\question
\begin{verbatim}
int num_threads = 4; 

omp_set_num_threads(num_threads); 

#pragma omp parallel 
{ 
    int start = omp_get_thread_num(); 
    for (int x = start; x < len; x+=4) { 
        *(A+x) = x; 
    } 
} 

\end{verbatim}

Is the code: \\
A) Always Incorrect \\
B) Sometimes Incorrect \\ 
C) Always Correct, Slower than Serial \\
D) Always Correct, Speed relative to Serial depends on Caching Scheme \\
E) Always Correct, Faster than Serial \\

\begin{solution}[0.5in]
D \\
Justification: This is similar to 3.3, but instead of assigning each thread its own chunk of data, the memory access of each thread is interleaving. For example, if the length of the array is 100, then thread 0 will modify memory indexed at 0, 4, 8, …, 96. In this mode, the memory address that is being accessed by each thread are relatively far away from each other. So it’s likely that threads can’t utilize the spatial locality of cache. 

\end{solution}
\end{blocksection}