\begin{blocksection}
\question
Optimize the following C code for factorial by using SIMD instructions.

\begin{verbatim}
double factorial(int n) {
    double f = 1.0;
    for (int i = 1 ; i <= n; i++) {
        f *= (double) i;
    }
    return f; 
}
\end{verbatim}

You will find the following SIMD instructions very useful.
\begin{verbatim}
// returns vector(s[0], s[1], s[2], s[3])
__m256d _mm256_loadu_pd(double *s);
// stores s[i] = v_i where i = 0, 1, 2, 3
void _mm256_store_pd(double *s, __m256d v);
// returns vector which is the element-wise product of a and b
__m256d _mm256_mul_pd(__m256d a, __m256d b);
\end{verbatim}

We provide the following skeleton code. Please fill in the blanks.

\begin{verbatim}
double factorial(int k) {
    int i, j;
    double f_init[] = {____, ____, ____, ____};
    double f_res[4];
    double f = 1.0;
    __m256d f_vec = ____________________________________________________________
    for (i = 1; i <= _________________________; _____________________________) {
        double l[] = {
        (double) _________________________, (double) __________________________,
        (double) _________________________, (double) __________________________};
        __m256d data = __________________________________________________________
        _________________________________________________________________________   
    }
    // reduce vector to array
    _mm256_store_pd(__________________________________________);
    // reduce array to value
    for (j = 0; j < 4; j++) {
        f = _____________________________________________________________________;
    }
    // handle tail case
    for ( ; i <= n; i++) {
        _______________________________
    }
    return f; 
}
\end{verbatim}

\end{blocksection}

\begin{solution}
\begin{verbatim}
    double f_init[] = {1.0, 1.0, 1.0, 1.0};
    __m256d f_vec = _mm256_loadu_pd(f_init);
    for (i = 1; i <= (n/4)*4; n += 4) {
        double l[] = {
        (double) i, (double) i+1,
        (double) i+2, (double) i+3};
        __m256d data = _mm256_loadu_pd(l);
        f_vec = _mm256_mul_pd(f_vec, data);   
    }
    _mm256_store_pd(f_res, f_vec);
    for (j = 0; j < 4; j++) {
        f = f * f_res[j];
    }
    for ( ; i <= n; i++) {
        f *= (double) i;
    }
\end{verbatim}
\end{solution}

