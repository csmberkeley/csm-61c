\begin{blocksection}
\question

You notice that floats can generally represent much larger numbers than integers, and decide to make a modified RISC instruction format in which all immediates for jump instructions are treated as 12-bit floating point numbers with a mantissa of 7 bits and with a standard exponent bias of 7.

\begin{parts}

\part
To jump the farthest, you set the float to be the most positive (not ∞) integer representable. What are those 12 bits (in hex)?
\begin{solution}[0.5in]
0x77F
\end{solution}

\part
What is the value of that float (in decimal)?
\begin{solution}[0.5in]
255
\end{solution}

\part
Between 0 and (b)’s answer(inclusive), how many integers are not representable?
\begin{solution}[0.5in]
0
\end{solution}

\end{parts}

\end{blocksection}