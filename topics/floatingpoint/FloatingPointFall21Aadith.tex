%need to add solutions for this question
For the following questions, we will be using the following breakdown for a 32-bit floating point number:
\newline
\textbf{1 sign bit}
\newline
\textbf{11 bit Exponent}
\newline
\textbf{20 bit Mantissa}
\newline

\newline
\begin{blocksection}
\question What is the bias of the exponent?
\newline \newline 
\begin{solution}
\begin{verbatim}
-(2^11/2 - 1) = -1023
\end{verbatim}
\end{solution}
\question Convert the floating point number 7.875 to hex?
\newline \newline \newline \newline \newline \newline \newline
\begin{solution}
\begin{verbatim}
7.875 = 4 + 2 + 1 + 0.5 + 0.25 + 0.125
=> 111.111 
=> 2^2 * 1.11111

sign = 0 because we are working with a positive number
exp + bias = 2 => exp - 1023 = 2 => exp = 1025 => 10000000001
mantissa/significand = 111110000000...0

Putting it all together we have the following 32 bits,
0 10000000001 11111000000000000000
which translates to the hex value
0x401F8000
\end{verbatim}
\end{solution}
\question How many NaNs can we represent?
\newline \newline \newline \newline \newline
\begin{solution}
\begin{verbatim}
There's 2^{20}−1 possible positive NaNs (we subtract one for infinity)
And there's 2^{20} - 1 possible negative NaNs (we subtract one for infinity again). 
Combined, that gives us 2^{21} - 2 possible NaN values
\end{verbatim}
\end{solution}
\question What is the largest positive number we can represent (answer in hex)?
\newline \newline \newline
\begin{solution}
\begin{verbatim}
0 11111111110 11111111111111111111
which is
0x7FEFFFFF
\end{verbatim}
\end{solution}




\end{blocksection}

