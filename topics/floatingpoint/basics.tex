\begin{blocksection}
\question

\begin{parts}
\part
How would $10.625$ be represented in floating point format?

\begin{solution}[0.5in]
$10.625 = 8 + 2 + \frac{1}{2} + frac{1}{8} → 1010.101_2$ in binary
$1010.101_2 = 1.010101_2 x 2^3$ → Exp $- 127 = 3$
Sign: $0$, Exponent: $130 = 10000010_2$, Significand: $010101_2$
\lstinline$0100 0001 0010 1010 0000 0000 0000 0000 = 0x412A0000$

\end{solution}

\part
What decimal number is encoded as \lstinline$0xC0A80000$?

\begin{solution}[0.5in]
\lstinline$0xC0A80000$ = $1100 0000 1010 1000 0000 0000 0000 0000_2$
Sign: $1$, Exponent: $100 0000 1_2= 129$, Significand: $0101_2$
$(-1) \times 1.0101_2 \times 2^{129 - 127} = -1.0101_2 \times 2^2 = -101.01_2 = -5.25$
\end{solution}

\part
How many non-negative floats are strictly less than 2?

\begin{solution}[0.5in]
Only possibility is if the exponent is within the range $[0, 127]$, as any value $> 127$ would make the float $>= 2$. Then, we can have any permutation of the significand without increasing the number by more than $1$. Thus, there are $27 \times 223 = 230$ floats $< 2$.
\end{solution}

\part
What is the smallest positive value that can be stored using a single precision float?

\begin{solution}[0.5in]
\lstinline$0x00000001$ = $2^{-23} * 2^{-126}$
\end{solution}

\end{parts}

\end{blocksection}